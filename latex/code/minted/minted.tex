% Copyright 2024 Hajun Park
%
% Licensed under the Apache License, Version 2.0 (the "License");
% you may not use this file except in compliance with the License.
% You may obtain a copy of the License at
%
%     http://www.apache.org/licenses/LICENSE-2.0
%
% Unless required by applicable law or agreed to in writing, software
% distributed under the License is distributed on an "AS IS" BASIS,
% WITHOUT WARRANTIES OR CONDITIONS OF ANY KIND, either express or implied.
% See the License for the specific language governing permissions and
% limitations under the License.

\documentclass[../../latex]{subfiles}

\begin{document}

\section{minted}

\mintinline{latex}{\usepackage{listings}}

Minted uses \href{https://pygments.org/}{Pygments} for syntax highlighting.

Install \href{https://python.org/}{Python} and then Pygments.

\begin{minted}{bash}
$ pip install Pygments
\end{minted}

To use Pygments on \LaTeX, you need to pass \mintinline{bash}{-shell-escape} flag to \LaTeX.

\begin{minted}{bash}
$ lualatex -shell-escape <file>
\end{minted}

If you want to compile \LaTeX document containing minted with Visual Studio Code and
\href{https://marketplace.visualstudio.com/items?itemName=James-Yu.latex-workshop}{LaTeX Workshop}
Plugin, add the following to \mintinline{text}{settings.json}.

\inputminted{json}{\currfiledir ex-settings.json}

\subsection{Inline code}

\mintinline{latex}{\mintinline{<language>}{<text>}}

\subsection{Code block}

For single line: \mintinline{latex}{\mint{<language>}{<text>}}

\begin{minipage}{0.47 \textwidth}
  \begin{minted}{latex}
\mint{python}{
print("Hello, world!")
}
  \end{minted}
\end{minipage}
\hfill
\begin{minipage}{0.47 \textwidth}
  \mint{python}{print("Hello, world!")}
\end{minipage}

For multiple lines: \mintinline{latex}{\begin{minted} ... \end{minted}}

\begin{minipage}{0.47 \textwidth}
  \begin{minted}[escapeinside=~~]{latex}
\begin{minted}{python}
def hello():
    print("Hello, world!")
\end{~minted~}
  \end{minted}
\end{minipage}
\hfill
\begin{minipage}{0.47 \textwidth}
  \begin{minted}{python}
def hello():
    print("Hello, world!")
  \end{minted}
\end{minipage}

\subsection{Input file}

\mintinline{latex}{\inputminted{<language>}{<file-path>}}

\begin{minipage}{0.47 \textwidth}
  \begin{minted}{latex}
\inputminted{python}{hello.py}
  \end{minted}
\end{minipage}
\hfill
\begin{minipage}{0.47 \textwidth}
  \inputminted{python}{\currfiledir hello.py}
\end{minipage}

\subsection{Captions and labels}

Minted provides floating listing environment to use with caption and label.

\begin{minipage}{0.47 \textwidth}
  \begin{minted}{latex}
\begin{listing}[H]
  \mint{python}|print("Hello, world!")|
  \caption{Code example}
  \label{lst:example}
\end{listing}
  \end{minted}
\end{minipage}
\hfill
\begin{minipage}{0.47 \textwidth}
  \begin{listing}[H]
    \mint{python}|print("Hello, world!")|
    \caption{Code example}
    \label{lst:example}
  \end{listing}
\end{minipage}

\subsection{Options}

\subsubsection{Setting global minted options}

inline \& code blocks

\begin{minted}{latex}
\setminted{<options>}
\setminted[<language>]{<options>}
\end{minted}

inline

\begin{minted}{latex}
\setmintedinline{<options>}
\setmintedinline[<language]{<options>}
\end{minted}

\subsubsection{Defining shortcuts}

minted environment

\begin{minted}{latex}
\newminted{<language>}{<options>} % default environment-name: <language>code
\newminted[<environment-name>]{<language>}{<options>}

\begin{<environment-name>}
\end{<environment-name>}
\end{minted}

mint command

\begin{minted}{latex}
\newmint{<language>}{<options>} % default macro-name: <language>
\newmint[<macro-name>]{<language>}{<options>}

\<macro-name>/<text>/ % ``/'' can be replaces by any character
\end{minted}

mintinline command

\begin{minted}{latex}
\newmintinline{<language>}{<options>} % default macro-name: <language>inline
\newmintinline[<macro-name>]{<language>}{<options>}

\<macro-name>/<text>/ % ``/'' can be replaces by any character
\end{minted}

inputminted command

\begin{minted}{latex}
\newmintedfile{<language>}{<options>} % default macro-name: <language>file
\newmintedfile[<macro-name>]{<language>}{<options>}

\<macro-name>{<file-path>}
\end{minted}

\subsubsection{Available options}

\begin{itemize}
  \item autogobble (boolean): Remove gobble (leading whitespace)
  \item breaklines (boolean): Automatically break long lines
  \item frame (none | leftline | topline | bottomline | lines | single): Put lines around the code
  \item linenos (boolean): Linen numbers
  \item numbersep (dimension): Gap between numbers and start of line
\end{itemize}

\begin{minted}{latex}
\setminted{
  autogobble,
  breakanywhere,
  breaklines,
  frame=single,
  linenos,
  numbersep=2mm,
}
\end{minted}

\end{document}
